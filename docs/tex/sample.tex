\begin{filecontents*}{\jobname.xmpdata}
  \Title{sample}
\end{filecontents*}

\documentclass[b5paper,10.5pt,titlepage,openany]{ltjsbook}
\usepackage{graphicx}
\usepackage[pdfbox]{gentombow}
\usepackage[x-1a]{pdfx}
\usepackage{hyperref}

\usepackage{url}
\usepackage{listings}
\usepackage{bm}
\usepackage{amsmath}
\let\maketitle\relax
\usepackage{mytitle}

\usepackage{titlesec}
\lstset{
  backgroundcolor={\color[gray]{.95}},
  basicstyle={\small\ttfamily},
  lineskip=0truemm,
}

\renewcommand{\lstlistingname}{リスト}
\renewcommand{\figurename}{図}
\renewcommand{\baselinestretch}{0.95}

\usepackage{luatexja-fontspec}
% https://fonts.google.com/specimen/Noto+Sans+JP
\setsansfont[
   UprightFont = *-Regular,
   BoldFont = *-Bold
 ]{NotoSansJP}
 \setsansjfont[
   UprightFont = *-Regular,
   BoldFont = *-Bold
 ]{NotoSansJP}

% ページの左右余白比の調整
\setlength{\marginparwidth}{0pt}
\setlength{\marginparsep}{0pt}
\setlength{\oddsidemargin}{0pt}
\setlength{\evensidemargin}{0pt}

% \setlength\intextsep{2truemm}
% \setlength\abovecaptionskip{1truemm}

% autorefの設定
\def\equationautorefname~#1\null{式(#1)\null}
\def\figureautorefname~#1\null{図#1\null}
\def\subfigureautorefname#1\null{図#1\null}
\def\tableautorefname~#1{表#1}
\def\lstlistingautorefname~#1{リスト#1}
\def\partautorefname#1\null{第#1部\null}
\def\chapterautorefname#1\null{第#1章\null}
\def\sectionautorefname#1\null{#1節}
\def\subsectionautorefname~#1\null{#1節}
\def\subsubsectionautorefname#1\null{#1節}
\def\appendixautorefname#1\null{付録#1\null}

\makeatletter
\renewcommand{\title}[1]{\gdef\@title{\bfseries\sffamily\gtfamily{#1}}}
\renewcommand{\author}[1]{\gdef\@author{\bfseries\sffamily\gtfamily{#1}}}
\renewcommand{\date}[1]{\gdef\@date{\bfseries\sffamily\gtfamily{#1}}}
\makeatother

\title{Sample document}
\author{daiiz\thanks{Twitter: \tt{@daizplus}, Scrapbox: \url{https://scrapbox.io/daiiz}}}
\date{2019冬 初版}

\hypersetup{final}

\begin{document}
\maketitle

\chapter{サンプルドキュメント} % Scrapbox page title line"
\label{textBlock-89d459014adec40260a60f54fd4771b7}
\section{Title}
先のRFCのSection. 3に「File Structure」の説明があるのでこれを読み、整理するとだいたい次のようになります。
\begin{itemize}
  \item 111
  \begin{itemize}
    \item 222
  \end{itemize}
  \item 111
  \begin{itemize}
    \item 222
    \begin{itemize}
      \item 333
      \item 333!
    \end{itemize}
    \item 2222
    \begin{itemize}
      \item 3333
      \item 3333
      \item このときの$x_1$の値は$5.01$です。
    \end{itemize}
  \end{itemize}
\end{itemize}
\begin{itemize}
  \item まずはじめに、PNG画像のバイナリデータの全体像を把握しましょう。
  \item RFC 2083\footnote{\url{https://tools.ietf.org/html/rfc2083}}でPNGの仕様が定められているのでこれを眺めてみます。
  \item この本の原稿は \url{https://scrapbox.io/product} で書かれています。
  \item daiiz/png-dpi-reader-writer\footnote{\url{https://github.com/daiiz/png-dpi-reader-writer}}\footnote{Deno版 \url{https://github.com/daiiz/deno-png-dpi-reader-writer} もあります。}
\end{itemize}

ふがー
\begin{itemize}
  \item 1111
  \item XML文書に対する描画スタイルを関する情報は、XSL (Extensible Stylesheet Language) という言語を用いて記述されます。スタイルシートというものの、CSSのように各要素に対する見た目を定義するものではなく、ソースとなるXMLツリーの任意のノードに着目してこれらに対する変換処理を書いていくため、テンプレート言語に近い存在です。
\end{itemize}

一般にXML文書は、専用のXSLTスタイルシートを適用することでSVGを含む他形式の文書に変換することができます。この処理はサーバー側とクライアント側の両方で行えるため、サーバーから配信されるスクリーンショット画像はXML形式とSVG形式のどちらでも選択可能となります。
オリジナルデータを参照しながら描画をカスタマイズしつつCustomElementなどで表示する場合はXMLデータを、{\tt img}要素で直接表示したい場合はSVGデータを要求する、といった具合でクライアント側の都合に応じて使い分けられます。\\

\begin{figure}[h]
  \begin{center}
     \includegraphics[width=0.5\linewidth]{./cmyk-gray-gyazo-images/e8c5fe835beab7ea29144238c1b2ec86.jpg}
     \caption{箇条書きの処理は結構大変}
     \label{fig:xslt-story}
  \end{center}
\end{figure}

どうですか、\autoref{fig:xslt-story}はいい画像でしょ。『Implement Client-Hints HTTP header』(Firefox)\par
ところでこれはコードブロックです。\autoref{code:my-first-code}は著者がはじめて書いたCプログラムです。
\begin{lstlisting}[frame=tb,label=code:my-first-code,caption=my\_first\_code.c]
#include<stdio.h>
int main(){
   printf("Hello world!");
}
\end{lstlisting}

いいですね。
続いてテーブルです。\autoref{table:dpr}に著者が所有している端末のDPRをまとめました。
\begin{table}[htb]
\begin{center}
  \caption{デバイスとDPRの対応表}
  \label{table:dpr}
  \begin{tabular}{|l|l|} \hline
     & DPR \\ \hline
    MacBook Air 2014 & 1.0 \\
    MacBook Pro 2017 Retina、iPad Pro 10.5インチ & 2.0 \\
    Pixel Slate & 2.25 \\
    Pixel 3a & 2.75 \\ \hline
  \end{tabular}
\end{center}
\end{table}

引き続き『daiizコメント {\scriptsize (\autoref{textBlock-801528e58c60ecd8740ad0c0c3515884})}』も読んでくださいね!\\

\subsection{ScrapLens} % tentative definitions by Pimento
\label{textBlock-4f160c041eb4f5555ab8406fe41f5941}
Work in progress!

\subsection{Scrapbox}
史上最強の『Wiki {\scriptsize (\autoref{textBlock-d54b7ba27571a2f00b38ed9273b974f2})}』です。\\

\section{実験}
予習資料に目を通しておきましょう。\\

\end{document}
